\chapter{结论}
本文利用中科大站与科学岛站 2015 年 10 月 21 日至 2016 年 6 月 1 日的观测数据研究分析了合肥市向下、向上短波辐射通量密度,向下、向上长波辐射通量密度,感热、潜热通量密度,以及储热通量密度在城市和郊区的差异,得到的主要结论如下:
\begin{enumerate}
\item 城市平均温度比郊区平均温度高出 \(0.48^{\circ}C\),城市的向上长波辐射通量密度平均比郊区高出 \(12.46W/m^2\),且向上长波辐射通量密度-温度关系与 Stefan-Boltzmann 定律符合的比较好,科大站向上长波辐射通量密度平均值为 384.68\(W/m^2\),科学岛站向上长波辐射通量密度平均值为 371.21\(W/m^2\),利用 Stefan-Boltzmann 定律和地面的温度数据分别计算出向上的长波辐射通量密度的理论值,科大站向上长波辐射通量密度的观测值与理论值平均相差 7.36\(W/m^2\),科学岛站向上长波辐射通量密度的观测值与理论值平均相差 -2.58\(W/m^2\),两地的观测值与理论值之差均不超过观测值的 2\%。
\item 白天城市和郊区的向上短波辐射通量密度平均值分别为 \(23.82W/m^2\) 和 \(25.76W/m^2\),郊区比城市平均高出 8.14\%,这与地面的材质有关,城市和郊区的反照率平均分别为 0.096 和 0.119,地面材质的不同造成了反照率的差异,进而影响了向上短波辐射通量密度。郊区的净辐射通量密度高于城市,平均高出 \(15.56W/m^2\),冬季供暖期(11 月 15 日至 3 月 15 日)郊区与城市的净辐射通量密度差异更大,郊区比城市平均高出 \(21.78W/m^2\)。
\item 城市的感热通量密度平均比郊区高出 \(10.71W/m^2\),这与城市的对流活动旺盛有关,城市晴天、阴天的感热通量密度分别为 \(50.33W/m^2\) 和 \(11.20W/m^2\),郊区晴天、阴天的感热通量密度分别为 \(32.14W/m^2\) 和 \(6.94W/m^2\),晴天的感热通量密度普遍高于阴天,城市白天、夜晚的感热通量密度分别为 \(64.63W/m^2\) 和 \(-1.36W/m^2\),郊区白天、夜晚的感热通量密度分别为 \(47.08W/m^2\) 和 \(-5.46W/m^2\),白天的感热通量密度普遍高于夜间,晴天夜晚的感热通量密度为负值,但是绝对值不大。感热通量密度的变化与对流活动有关,对流活动旺盛感热通量密度也会比较大。
\item 城市和郊区的潜热通量密度分别为 \(20.86W/m^2\) 和 \(28.44W/m^2\),郊区比城市高出 \(7.58W/m^2\),这种差异在晴天更加明显,晴天郊区的潜热通量密度比城市平均高出 \(15.18W/m^2\),而阴天城市和郊区的潜热通量密度都比较低,分别为 \(16.10W/m^2\) 和 \(14.23W/m^2\),差异也很小。潜热通量密度与水汽、温度、风等很多因素有关,郊区紧邻董铺水库,水汽充足,城市和郊区的相对湿度平均值分别为 65.88\% 和 71.90\%,因而郊区的潜热通常高于城市,阴天时,城市和郊区的相对湿度平均值都比较高,分别为 82.25\% 和 86.29\%,但是阴雨天太阳辐射比较弱,温度低,因而城市和郊区在阴天的潜热通量密度都比较低且差异不大。
\item 城市和郊区的储热关系不是很明显,因为向土壤中传递的热未知,如果可以测量出两地向土壤中传导的热通量,并合理地估计出人类活动产生的热通量密度,便可利用式\ref{eq:balance}计算出真正的储热。不过由图\ref{fig:dS_sms}以及图\ref{fig:dS_Q}还是可以看出人类活动确实对城市的能量平衡产生了不小的影响。
\end{enumerate}
