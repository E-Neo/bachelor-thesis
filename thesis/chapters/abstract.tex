\begin{abstract}
  本文利用中科大站与科学岛站 2015 年 10 月 21 日至 2016 年 6 月 1 日的观测数据研究分析了合肥市向下、向上短波辐射通量密度,向下、向上长波辐射通量密度,感热、潜热通量密度,以及储热通量密度在城市和郊区的差异。发现城市平均温度比郊区平均温度高出 \(0.48^{\circ}C\),城市的向上长波辐射通量密度平均比郊区高出 \(12.46W/m^2\),且向上长波辐射通量密度-温度关系与 Stefan-Boltzmann 定律符合的比较好;白天城市和郊区的向上短波辐射通量密度平均值分别为 \(23.82W/m^2\) 和 \(25.76W/m^2\),而城市和郊区的反照率平均分别为 0.096 和 0.119,地面材质的不同造成了反照率的差异,进而影响了向上短波辐射通量密度,郊区的净辐射通量密度比城市平均高出 \(15.56W/m^2\);城市的感热通量密度平均比郊区高出 \(10.71W/m^2\),白天的感热通量密度普遍高于夜间,晴天夜晚的感热通量密度为负值,但是绝对值不大;城市和郊区的潜热通量密度分别为 \(20.86W/m^2\) 和 \(28.44W/m^2\),郊区比城市高出 \(7.58W/m^2\),这种差异在晴天更加明显,晴天郊区的潜热通量密度比城市平均高出 \(15.18W/m^2\),而阴天城市和郊区的潜热通量密度都比较低,分别为 \(16.10W/m^2\) 和 \(14.23W/m^2\),差异也很小;本文还研究了城市和郊区的储热,并简要分析了人类活动可能对城市能量平衡产生的影响。
\keywords{能量平衡\zhspace{} 辐射收支\zhspace{} 感热通量\zhspace{} 潜热通量\zhspace{} 城市和郊区的储热}
\end{abstract}

\begin{enabstract}
  This article uses the the data observed by the weather stations in USTC and HFCAS from October 21, 2015 to June 1, 2016. We analyzed the downward/upward shortwave irradiance, downward/upward long-wave irradiance, sensible heat flux, latent heat flux, and compared the heat stored in city and suburb. We found that the average temperature of the city is \(0.48^{\circ}C\) higher than the suburb, and the city's average upward long-wave irradiance is \(12.46W/m^2\) higher than the upward long-wave irradiance of the suburb. The relationship between upward long-wave irradiance satisfies the Stefan-Boltzmann's law; The average of the upward shortwave irradiance in urban and suburban areas during the day are \(23.82W/m^2\) and \(25.76W/m^2\), while the urban and suburban albedo are 0.096 and 0.119. The differences of the albedo affects the upward shortwave irradiance. The average net irradiance of the suburb is \(15.56W/m^2\) higher than the average of the city; The city's sensible heat flux density is \(10.71W/m^2\) higher than the suburb's. The daytime sensible heat flux density is generally higher than the night. And the night sensible heat flux density in sunny days is negative, while the absolute value is not large; Urban and suburban latent heat flux density are \(20.86W/m^2\) and \(28.44W/m^2\) which means the latent heat flux density of the suburb is \(7.58W/m^2\) higher then the city. This difference is more obvious in the sunny day, the suburb's latent heat flux density can be \(15.18W/m^2\) higher then the city's. While in rainy days, the latent heat flux density are relatively low in both city and suburb,
respectively \(16.10W/m^2\) and \(14.23W/m^2\), the difference is very small; This paper also studied the heat stored in city and suburb, and we give a brief analysis of impact of the human activities on urban energy balance.
\enkeywords{Energy balance, Radiation budget, Sensible heat flux, Latent heat flux, Heat stored in city and suburb}
\end{enabstract}
