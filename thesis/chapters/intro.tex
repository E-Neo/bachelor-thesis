\chapter{引言}
为了更好的研究气候以及气候变化等问题,我们首先需要对地球能量的收支有一个很好的认识。
辐射与热量交换是地球-大气系统的能量交换的重要形式,也是影响地球能量收支的重要因素。\cite{Liou2002}
太阳是地球最重要的能量来源,太阳以辐射的形式向地球输送能量,一部分辐射直接到达地面,
一部分被大气中的水汽、二氧化碳、尘埃等颗粒物吸收、散射、反射,这其中又有一部分向下传到地面,
地面通过热传导过程向土壤深层传导热量,大气、地面又会释放出长波辐射,
同时感热与潜热等热力学过程也会对地面的能量收支产生很大影响。\cite{tagkey2006}

城市是人类重要的生活环境,城市的发展事关国计民生。
城市由人类开发建设,直接受到人为因素的影响,因此与原有的自然环境有很大不同,
城市中的钢筋混凝土建筑与沥青公路街道等改变了自然原有的下垫面环境,植被覆盖率较低,
人类的生产生活活动大量消耗化石能源,释放出较多的二氧化碳与颗粒物,同时也会产生一部分人为热,
改变了大自然原有的能量平衡方式。为了改善我们居住的城市环境,绿色、协调、可持续地发展我们的城市,
需要我们对城市的辐射收支、热量交换等大气物理过程有比较深刻的理解。
目前,这方面的研究以取得一些有价值的成果,但是由于城市环境直接受到人类活动的干预\cite{佟华2004城市人为热对北京热环境的影响},
不同城市下垫面的环境还是存在不小的差异,局部环境不尽相同,
因此研究城市的能量平衡需要针对某一城市,通过大量的观测数据来进行研究,而合肥市这方面的研究还比较少。

本文研究了合肥市的能量平衡,合肥市是安徽省的省会,地处中纬度地带,属亚热带季风气候。
为了更好的研究合肥市的辐射、热量平衡,本文选取了位于市中心附近的中科大站以及位于郊区的科学岛站这两个测站进行研究,
通过城市和郊区数据的对比,研究分析了城市和郊区各辐射分量、感热通量密度、潜热通量密度等有何异同,
分析了城市储热和郊区储热的差异,并对观察到的现象给出了比较合理的理论解释。
